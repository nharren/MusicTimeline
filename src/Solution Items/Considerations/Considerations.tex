\documentclass[letterpaper]{report}
\usepackage[T1]{fontenc}
\usepackage[utf8]{inputenc}
\usepackage[margin=2cm]{geometry}
\usepackage[colorlinks=true, linkcolor=blue]{hyperref}
\usepackage{titlesec}
\author{Nathan Harrenstein}
\title{Composer Timeline Considerations}

\renewcommand{\arraystretch}{1.75}

\titleformat{\section}{\normalfont\fontsize{11}{13.2}\bfseries}{\thesection}{1em}{}
\titlespacing{\section}{0pt}{18pt plus 6pt minus 6pt}{18pt plus 6pt minus 6pt}

\begin{document}

\maketitle

\renewcommand{\baselinestretch}{1.25}\normalsize
\tableofcontents
\renewcommand{\baselinestretch}{1}\normalsize

\chapter{Introduction}

This document serves as a historical record of the design decisions of every aspect of the Composer Timeline application. The intent of this document is threefold: to avert any consequences which may arise due to unforeseen conditions of real-world data or usages, to provide a resource should design discussions arise similar to those already discussed, and to ensure that decisions are not made hastily and uninformed. 

\chapter{Platform Considerations}

\section{There are many different ways of storing data.}
\label{P.1}

\begin{tabular}{ p{0.1\linewidth} p{0.825\linewidth} }
  Examples:  &  \\ 
  Resolution:  &  \\
  Rationale:  &  \\
  Status:  &  \\
  Related:  & 
\end{tabular}

\section{There are many suitable programming languages.}
\label{P.2}

\begin{tabular}{ p{0.1\linewidth} p{0.825\linewidth} }
  Examples: & C++, Java, HTML + Javascript, VB, CS, Python, Ruby, etc. \\ 
  Resolution: & Use CS leveraging WPF. \\
  Rationale: & This is the platform I know best, and the time spent learning the ins and outs of a new language would make this project far more difficult. \\
  Status: & Complete.
\end{tabular}

\section{Applications can be built for the internet or an operating system.}
\label{P.3}

\begin{tabular}{ p{0.1\linewidth} p{0.825\linewidth} }
  Examples: & Google Docs vs. Microsoft Office \\ 
  Resolution: & Stick with the desktop for now. \\
  Rationale: & The intent of this application is to be able to interface with a user's music library, and it requires the filesystem to store a number of components (e.g. imported music, composer images, flags, favicons, etc.). \\
  Status: & Complete.
\end{tabular}

\chapter{Data Considerations}

\section{Catalog numbers can be assigned to either composition collections or compositions.}
\label{D.1}

\begin{tabular}{ p{0.1\linewidth} p{0.825\linewidth} }
  Examples: & The composer Joseph Haydn has a collection of String Quartets, Op. 20 and Robert Schumann has a Piano Sonata, Op. 22. \\ 
  Resolution: & Implement two catalog number tables, one for compositions and another for composition collections, called ``CompositionCollectionCatalogNumber'' and ``CompositionCatalogNumber''. \\
  Rationale: & Creating two tables in the database instead of one singular ``CatalogNumber'' table increases query speed. \\
  Status: & Complete. \\
  Related: & \hyperref[UI.1]{Catalog numbers can be assigned to either composition collections or compositions.} (User Interface Correlation)
\end{tabular}

\section{Composition numbers from different catalogs can be assigned to the same composition or composition collection.}
\label{D.2}

\begin{tabular}{ p{0.1\linewidth} p{0.825\linewidth} }
  Examples: & The composer Joseph Haydn has a collection of String Quartets, Op. 20, which also have Hoboken numbers (Hob. III:31--36). \\ 
  Resolution: & The database will contain ``CompositionCatalog'', ``CompositionCatalogNumber'', and ``CompositionCollectionCatalogNumber'' tables. The ``Composer'' table will have a one to many relationship with the ``CompositionCatalog'' table, and the  ``CompositionCatalog'' table will have a one to many relationship with the ``CompositionCatalogNumber'' and ``CompositionCollectionCatalogNumber'' tables. \\
  Rationale: & Besides satisfying the case, these new tables reduce data redundancy and couple the composition catalog with the composer whose works were cataloged. The reduction of data redundancy occurs because the ``CatalogNumber'' table will reference a composition catalog which will store that composition catalog's prefix. Therefore, the prefix will not have to be repeatedly entered for each catalog number. \\
  Status: & Complete. \\
  Related: & \hyperref[UI.2]{Composition numbers from different catalogs can be assigned to the same composition or composition collection.} (User Interface Correlation); \hyperref[D.3]{Opus numbers and catalog numbers are different types of data.} (Data Descendant; \hyperref[D.4]{Beethoven has a catalog of works without opus numbers.} (Data Descendant)
\end{tabular}

\section{Opus numbers and catalog numbers are different types of data.}
\label{D.3}

\begin{tabular}{ p{0.1\linewidth} p{0.825\linewidth} }
  Examples: & Catalog numbers are collected in composition catalogs, but opus numbers have no referable catalog. \\ 
  Resolution: & Treat opus numbers as a type of catalog number from an unnamed composition catalog. \\
  Rationale: & According to Wikipedia, ``In the eighteenth century, publishers usually assigned opus numbers when publishing groups of like compositions\ldots[f]rom about 1800, composers, especially Beethoven, assigned an opus number to a work, and later to a set of works, especially songs and short piano pieces''. In a sense, opus numbers \textit{are} a way of cataloging a composer's compositions, just as other composition catalogs. The difference is that opus numbers derive from a publisher or a composer whereas a composition catalog derives from a musicologist. Because the goal of both Opus numbers and catalog numbers is to assign a unique identifier to a composer's composition, it makes sense to treat them as the same data. Because Opus numbers do not derive from a referable, named catalog of compositions, we treat them as entries of a nameless catalog. \\
  Status: & Complete. \\
  Related: & \hyperref[D.2]{Composition numbers from different catalogs can be assigned to the same composition or composition collection.} (Data Ancestor)
\end{tabular}

\section{Beethoven has a catalog of works without opus numbers.}
\label{D.4}

\begin{tabular}{ p{0.1\linewidth} p{0.825\linewidth} }
  Examples: & Beethoven's Minuet for Orchestra in E flat major is assigned the catalog number WoO 3. \\ 
  Resolution: & Treat ``WoO'' as a prefix for the ``Kinsky-Halm'' composition catalog. \\
  Rationale: & According to Wikipedia, ``Werke ohne Opuszahl (``Works without opus number'') (WoO), also Kinsky–Halm Catalogue, is a German musical catalogue prepared in 1955 by Georg Kinsky and Hans Halm, listing all of the compositions of Ludwig van Beethoven that were not originally published with an opus number, or survived only as fragments.'' \\
  Status: & Complete. \\
  Related: & \hyperref[D.2]{Composition numbers from different catalogs can be assigned to the same composition or composition collection.} (Data Ancestor)
\end{tabular}

\section{A composition or composition collection can be revised.}
\label{D.5}

\begin{tabular}{ p{0.1\linewidth} p{0.825\linewidth} }
  Examples: & Sergei Prokofiev's Symphony No. 4 exists in two forms: the original version (Op. 47), and the revised version (Op. 112). While Prokofiev gave his revisions different Opus numbers, most other composers do not. \\ 
  Resolution: & When naming a revised composition, put the name of the composition, followed by the common way of referring to that revision (e.g. Symphony No. 2 (Revised Version), Symphony No. 2 (1908 Version), Symphony No. 2 (1908 Revision), Symphony No. 2 (Version 1), etc.)  \\
  Rationale: & The name should uniquely identify a composition or composition collection. If the revision data were to be placed elsewhere, it would be difficult to tell at a glance how compositions with the same name and composer differ. \\
  Status: & Complete. \\
\end{tabular}

\section{A composition or composition collection might have been arranged for another instrument by one or multiple arrangers.}
\label{D.5}

\begin{tabular}{ p{0.1\linewidth} p{0.825\linewidth} }
  Examples: & Pictures at an Exhibition, a suite of ten piano pieces by Modest Mussorgsky was arranged for orchestra by Maurice Ravel. \\ 
  Resolution: & Put the arranger in the name of the composition or composition collection (e.g. Pictures at an Exhibition (arr. Maurice Ravel), Pictures at an Exhibition (arr. Maurice Ravel and Ludwig van Beethoven), etc)).  \\
  Rationale: & The composition should still reside with the composer, as it still contains the musical intention of the composer. It also needs to have a name which uniquely identifies the composition or composition collection. Also, if an arranger table were to be created in the future, it would be easy to tell which compositions were arrangements, therefore making lighter work. \\
  Status: & Complete. \\
\end{tabular}

\section{A composition or composition collection might have multiple composers.}
\label{D.6}

\begin{tabular}{ p{0.1\linewidth} p{0.825\linewidth} }
  Examples: & In 1937, Arthur Honegger and Jacques Ibert wrote the opera L'Aiglon. Ibert wrote Acts 1 and 5, Honegger the rest. \\ 
  Resolution: & Add a ``CompositionCollectionComposer'' and ``CompositionComposer'' table to the database. Both would contain a reference to a composer and a composition collection or composition (depending on the table).  \\
  Rationale: & The new tables would serve as junction tables between ``Composition'' and ``Composer'' and ``CompositionCollection'' and ``Composer'', so both relationships would be many to many. \\
  Status: & \textbf{Incomplete.} \\
  Related: & \hyperref[UI.6]{A composition or composition collection might have multiple composers.} (User Interface Correlation)
\end{tabular}

\section{There is much more real world data on classical music than is necessary or practical for the purposes of this application.}
\label{D.7}

\begin{tabular}{ p{0.1\linewidth} p{0.825\linewidth} }
  Examples: & Publishers, Album Labels, Recording Engineers, Recording Producers, Composition Catalog Publish Date, etc. \\ 
  Resolution: & Create a list of all data required by the timeline, data which is convenient, and data which might be used at a later time. \\
  Rationale: & Then it is known as well as possible, what data is required, and what could be added in at a later time. \\
  Status: & \textbf{Incomplete}.
\end{tabular}

\section{A composer name might be used ``firstname lastname'' for display or ``lastname, firstname'' for sorting.}
\label{D.8}

\begin{tabular}{ p{0.1\linewidth} p{0.825\linewidth} }
  Examples: & ``Johann Sebastian Bach'' or ``Bach, Johann Sebastian'' \\ 
  Resolution: & Have the user input the composer's sort-name. \\
  Rationale: & The display-name can be easily generated from the sort-name. \\
  Status: & \textbf{Incomplete}. \\
  Related: & \hyperref[D.9]{A composer can have multiple given names or surnames.} (Data Corollary); \hyperref[D.10]{A composer can have no surnames.} (Data Corollary); \hyperref[D.11]{A composer might be traditionally referred to by their firstname and lastname; their firstname, middlename, and last name; or their middlename and lastname.} (Data Corollary)
\end{tabular}

\section{A composer can have multiple given names or surnames.}
\label{D.9}

\begin{tabular}{ p{0.1\linewidth} p{0.825\linewidth} }
  Examples: & ``Johann Sebastian Bach'' or ``Carl Phillip Emanuel Bach'' are examples of multiple given names. ``Ralph Vaughn Williams'' is an example of an unhyphenated double-barrelled surname (Vaughn Williams).  \\ 
  Resolution: & Have the user input the composer's sort-name. \\
  Rationale: & Therefore, the user does not have to specify which names are given names or surnames--only how it should be displayed for sorting. \\
  Status: & \textbf{Incomplete} \\
  Related: & \hyperref[D.8]{A composer name might be used ``firstname lastname'' for display or ``lastname, firstname'' for sorting.} (Data Corollary); \hyperref[D.10]{A composer can have no surnames.} (Data Corollary); \hyperref[D.11]{A composer might be traditionally referred to by their firstname and lastname; their firstname, middlename, and last name; or their middlename and lastname.} (Data Corollary) \\
\end{tabular}

\section{A composer can have no surnames.}
\label{D.10}

\begin{tabular}{ p{0.1\linewidth} p{0.825\linewidth} }
  Examples: & ``Hafliði Hallgrímsson'' is an Icelandic name, which typically use patronyms (in this case ``Hallgrímsson'') rather than surnames.  \\ 
  Resolution: & Have the user input the composer's sort-name. \\
  Rationale: & Therefore, the user does not have to specify which names are given names or surnames--only how it should be displayed for sorting. \\
  Status: & \textbf{Incomplete}\\
  Related: & \hyperref[D.8]{A composer name might be used ``firstname lastname'' for display or ``lastname, firstname'' for sorting.} (Data Corollary); \hyperref[D.9]{A composer can have multiple given names or surnames.} (Data Corollary); \hyperref[D.11]{A composer might be traditionally referred to by their firstname and lastname; their firstname, middlename, and last name; or their middlename and lastname.} (Data Corollary) \\
\end{tabular}

\section{A composer might be traditionally referred to by their firstname and lastname; their firstname, middlename, and last name; or their middlename and lastname. }
\label{D.11}

\begin{tabular}{ p{0.1\linewidth} p{0.825\linewidth} }
  Examples: & ``Johann Sebastian Bach'', ``Troyal Garth Brooks'', or ``Johannes Brahms'' . \\ 
  Resolution: & Have the user input the composer's sort-name. \\
  Rationale: & Therefore, the user does not have to specify which names are given names, middle names, or surnames--only how it should be displayed for sorting. \\
  Status: & \textbf{Incomplete} \\
  Related: & \hyperref[D.8]{A composer name might be used ``firstname lastname'' for display or ``lastname, firstname'' for sorting.} (Data Corollary); \hyperref[D.9]{A composer can have multiple given names or surnames.} (Data Corollary); \hyperref[D.10]{A composer can have no surnames.} (Data Corollary) \\
\end{tabular}

\section{A composition collection name or composition name can be generated from data or input entirely by the user.}
\label{D.12}

\begin{tabular}{ p{0.1\linewidth} p{0.825\linewidth} }
  Examples: & ``Johann Sebastian Bach'' or ``Carl Phillip Emanuel Bach'' \\ 
  Resolution: & Have separate columns for firstname and lastname in the database. \\
  Rationale: & Both forms of names can then easily be made. \\
  Status: & \textbf{Incomplete}.
\end{tabular}

\section{The order of compositions within a composition collection can be identified by the catalog number.}
\label{D.12}

\begin{tabular}{ p{0.1\linewidth} p{0.825\linewidth} }
  Examples: & ``Johann Sebastian Bach'' or ``Carl Phillip Emanuel Bach'' \\ 
  Resolution: & Have separate columns for firstname and lastname in the database. \\
  Rationale: & Both forms of names can then easily be made. \\
  Status: & \textbf{Incomplete}.
\end{tabular}

\section{A key is used in a composition title to distinguish it from another work by the same name.}

\begin{tabular}{ p{0.1\linewidth} p{0.825\linewidth} }
  Examples: & ``Etude in E-flat major'', ``Etude in G major'', or ``Etude in A minor''. \\ 
  Resolution: & There will be no separate key field. Key will be included in a composition's common name when it is common to do so. \\
  Rationale: & A key was used in composition titles as a way of distinguishing works by the same name (source: \url{http://music.stackexchange.com/questions/6688/why-is-the-key-included-in-classical-music-titles}). Therefore, if a composition had a key, it does not guarantee that it would be included in the title. \\
  Status: & \textbf{Incomplete.}
\end{tabular}

\section{A flag path can be stored in the database or hard-coded into a switch statement.}

\begin{tabular}{ p{0.1\linewidth} p{0.825\linewidth} }
  Examples: & N/A \\ 
  Resolution: & Determine flag by naming a flag image the same as the nationality, then append the path and extension in code. \\
  Rationale: & Hardcoding a path in the database for every nationality would make changing the flag image directory difficult. Naming the flag image the same as its corresponding nationality would prevent the need for a switch statement/dictionary in code. \\
  Status: & Complete.
\end{tabular}

\section{A composer can be considered part of multiple eras.}

\begin{tabular}{ p{0.1\linewidth} p{0.825\linewidth} }
  Examples: & ``Ludwig van Beethoven'' is considered a Classical and Romantic composer. \\ 
  Resolution: & Create a ComposerEra table which links a composer to an era. \\
  Rationale: & A composer can then be linked to multiple eras. \\
  Status: & Complete.
\end{tabular}

\section{A composer can have multiple nationalities.}

\begin{tabular}{ p{0.1\linewidth} p{0.825\linewidth} }

Examples: & 

``George Frideric Handel'' is considered to be both English and German. \\ 
  
Resolution: & 

Create a ComposerNationality table which links a composer to a nationality. A composer can then be linked to multiple nationalities. \\
  
\end{tabular}

\section{A date might be unknown, approximate, or within a range of dates.}

\begin{tabular}{ p{0.1\linewidth} p{0.825\linewidth} }

Examples: & 

William Byrd has a birth date variously given as c.1539/40 or 1543. \\ 
  
Solution: & 

Using the ExtendedDateTimeFormat library, approximate dates, unknown dates, dates which fall within a known set or range, etc. can all be input. \\
  
\end{tabular}

\section{Ranges of dates can be stored in one column or in multiple columns, one for each end of the range.}

\begin{tabular}{ p{0.1\linewidth} p{0.825\linewidth} }

Examples: & 

Birth date and death date can be stored separately in ``BirthDate'' and ``DeathDate'' columns, or they can be stored together in a ``Dates'' column. \\ 
  
Solution: & 

While both have merits, birth and death dates are stored together in a ``Dates'' column. Having a single string is more efficient because it avoids the need to manipulate the data before it is parsed into an ExtendedDateTimeFormat object. \\
  
\end{tabular}

\chapter{Programming Considerations}

\chapter{User Interface Considerations}

\section{Catalog numbers can be assigned to either composition collections or compositions.}
\label{UI.1}

\begin{tabular}{ p{0.1\linewidth} p{0.825\linewidth} }
  Example: & The composer Joseph Haydn has a collection of String Quartets, Op. 20 and Robert Schumann has a Piano Sonata, Op. 22. \\ 
  Resolution: & In the ``AddFilePage'', both the composition collection and composition sections will have fields for catalog number. If a composition collection is given a catalog number, then compositions in that collection will have their catalog number fields pre-filled with the catalog number of the composition collection along with `` No. '' after which the user can enter in the number of the composition in the collection. The pre-filled content will be marked read-only. \\ 
  Rationale: & Pre-filling the composition's catalog number fields when a catalog number is assigned to the collection assures a consistent naming scheme for compositions in a collection (one could write, for example, Op. 2/1, Op. 2 No. 1, Op 2:1, etc.). Making the pre-filled portion read-only helps the user recognize not to deviate from the standard form. \\
  Status: & \textbf{Incomplete}. \\
  Related: & \hyperref[D.1]{Catalog numbers can be assigned to either composition collections or compositions.} (Data Correlation)
\end{tabular}

\section{Composition numbers from different catalogs can be assigned to the same composition or composition collection.}
\label{UI.2}

\begin{tabular}{ p{0.1\linewidth} p{0.825\linewidth} }
  Example: & The composer Joseph Haydn has a collection of String Quartets, Op. 20, which also have Hoboken numbers (Hob. III:31--36). \\ 
  Resolution: & In the ``AddFilePage'', you will be able to add composition catalogs (e.g. Bach-Werke-Verzeichnis) and assign them prefixes (e.g. BWV) in the composer section. The application then puts those prefixes along with ``Op.'' into a drop-down list in the composition and composition collection sections. To the right of the drop-down list is a text box in which the user can input the catalog number. The text box is always tied to the selected prefix; if there is already a value in the text box, and a user selects a different prefix in the drop-down list, the value of the previous prefix is stored in memory, and the text box is updated to reflect the value of the catalog number with the current prefix. \\
  Rationale: & The UI will satisfy the requirements, while having a minimal visual footprint and reducing the amount of redundant data entered in by the user. \\
  Status: & \textbf{Incomplete}. \\
  Related: & \hyperref[D.2]{Composition numbers from different catalogs can be assigned to the same composition or composition collection.} (Data Correlation)
\end{tabular}

\section{A composition or composition collection might have multiple composers.}
\label{UI.3}

\begin{tabular}{ p{0.1\linewidth} p{0.825\linewidth} }
  Examples: & In 1937, Arthur Honegger and Jacques Ibert wrote the opera L'Aiglon. Ibert wrote Acts 1 and 5, Honegger the rest. \\ 
  Resolution: & Change ``Composer'' section to ``Composer(s)'' section and include a list box of all composers in the database. A user can add composers, remove composers, and select multiple composers. When a single composer is selected in the list, all of the fields in the composer section update with the information for that composer. When multiple composers are selected, all of the fields in the composer section become disabled, but are filled with the values of all the composers selected delimited by semicolons.  \\
  Rationale: & This solves the problem, and it also makes it possible to easily add the information of the composers who were influences of the current composer. \\
  Status: & \textbf{Incomplete}. \\
  Related: & \hyperref[D.6]{A composition or composition collection might have multiple composers.} (Data Correlation)
\end{tabular}

\chapter{Validation Considerations}

\section{A composition or composition collection can be incomplete or unfinished.}
\label{V.1}

\begin{tabular}{ p{0.1\linewidth} p{0.825\linewidth} }
  Examples:  &  \\ 
  Resolution:  &  \\
  Rationale:  &  \\
  Status:  &  \\
  Related:  & 
\end{tabular}

\section{A composer can have no death date or location.}
\label{V.2}

\begin{tabular}{ p{0.1\linewidth} p{0.825\linewidth} }
  Examples:  &  \\ 
  Resolution:  &  \\
  Rationale:  &  \\
  Status:  &  \\
  Related:  & 
\end{tabular}

\section{A user can enter in multiple composers at once.}
\label{V.3}

\begin{tabular}{ p{0.1\linewidth} p{0.825\linewidth} }
  Examples:  &  \\ 
  Resolution:  &  \\
  Rationale:  &  \\
  Status:  &  \\
  Related:  & 
\end{tabular}

\end{document}